%! func: `font-setup'
%! meta: @{plainTeX} @{LuaTeX}

\ifx\fntmac\fmtversion\endinput\else\let\fntmac=\fmtversion\fi

\catcode`@=11

\adjustspacing=2
\protrudechars=2
\def\sethzcode#1{%
 \expandglyphsinfont#1 20 20 10}
\def\setprotcode#1{%
 \rpcode#1`\!=200
 \rpcode#1`\,=700
 \rpcode#1`\-=700
 \rpcode#1`\.=700
 \rpcode#1`\;=500
 \rpcode#1`\:=500
 \rpcode#1`\?=200
 \lpcode#1`\`=700
 \rpcode#1`\'=700
 \lpcode#1 92=500  % ``
 \rpcode#1 34=500  % ''
 \rpcode#1 123=300 % --
 \rpcode#1 124=200 % ---
 \rpcode#1`\)=50
 \rpcode#1`\A=50
 \rpcode#1`\F=50
 \rpcode#1`\K=50
 \rpcode#1`\L=50
 \rpcode#1`\T=50
 \rpcode#1`\V=50
 \rpcode#1`\W=50
 \rpcode#1`\X=50
 \rpcode#1`\Y=50
 \rpcode#1`\k=50
 \rpcode#1`\r=50
 \rpcode#1`\t=50
 \rpcode#1`\v=50
 \rpcode#1`\w=50
 \rpcode#1`\x=50
 \rpcode#1`\y=50
 \lpcode#1`\(=50
 \lpcode#1`\A=50
 \lpcode#1`\J=50
 \lpcode#1`\T=50
 \lpcode#1`\V=50
 \lpcode#1`\W=50
 \lpcode#1`\X=50
 \lpcode#1`\Y=50
 \lpcode#1`\v=50
 \lpcode#1`\w=50
 \lpcode#1`\x=50
 \lpcode#1`\y=0}

\expandafter\ifx\csname ccf\endcsname\relax\endinput\fi

\def\hex@#1{\ifcase#1 0\or 1\or 2\or 3\or 4\or 5\or 6\or 7\or 8\or 9\or A\or B\or C\or D\or E\or F\fi}

\font\textrm=ccr10 % roman
\font\textit=ccti10 % italic
\font\textsl=ccsl10 % slanted
\font\textbf=cmbx10 % bold
\font\textsc=cccsc10 % small caps
\font\oldsty=ccmi10 % old style numbers
\font\mathtext=ccr10
 \font\mathsubtext=ccr7
 \font\mathsubsubtext=ccr5
\font\mathlet=eurm10 \fontdimen2\mathlet=0pt
 \font\mathsublet=eurm7 \fontdimen2\mathsublet=0pt
 \font\mathsubsublet=eurm5 \fontdimen2\mathsubsublet=0pt
 \skewchar\mathlet='177 \skewchar\mathsublet='177 \skewchar\mathsubsublet='177
\font\mathsym=cmsy10
 \font\mathsubsym=cmsy7
 \font\mathsubsubsym=cmsy5
 \skewchar\mathsym='60 \skewchar\mathsubsym='60 \skewchar\mathsubsubsym='60
\font\mathext=cmex10
 \font\mathsubext=cmex10
 \font\mathsubsubext=cmex10
\font\mathscr=euxm10\fontdimen2\mathscr=0pt
 \font\mathsubscr=euxm7 \fontdimen2\mathsubscr=0pt
 \font\mathsubsubscr=euxm5 \fontdimen2\mathsubsubscr=0pt
 \skewchar\mathscr='60 \skewchar\mathsubscr='60 \skewchar\mathsubsubscr='60
\font\mathfr=eufm10\fontdimen2\mathfr=0pt
 \font\mathsubfr=eufm7 \fontdimen2\mathsubfr=0pt
 \font\mathsubsubfr=eufm5 \fontdimen2\mathsubsubfr=0pt
\font\matheuex=euex10\fontdimen2\matheuex=0pt
\font\eulerbf=eurb10 \fontdimen2\eulerbf=0pt % as needed
\font\teneq=cmr10 % equal sign

\def\rm{\fam\z@\textrm}
\def\it{\fam\itfam\textit} \textfont\itfam=\textit % family 4
\def\sl{\fam\slfam\textsl} \textfont\slfam=\textsl % family 5
\def\bf{\fam\bffam\textbf} \textfont\bffam=\textbf % family 6

\newfam\scrfam \edef\scrfam@{\hex@\scrfam}
\def\scr{\fam\scrfam }
\newfam\frfam \edef\frfam@{\hex@\frfam}
\def\frak{\fam\frfam }
\newfam\euexfam \edef\euexfam@{\hex@\euexfam}
\newfam\eqfam \edef\eqfam@{\hex@\eqfam}

\textfont0=\mathtext
 \scriptfont0=\mathsubtext
 \scriptscriptfont0=\mathsubsubtext
\textfont1=\mathlet \let\tfont=\teni
 \scriptfont1=\mathsublet
 \scriptscriptfont1=\mathsubsublet
\textfont2=\mathsym
 \scriptfont2=\mathsubsym
 \scriptscriptfont2=\mathsubsubsym
\textfont3=\mathext
 \scriptfont3=\mathsubext
 \scriptscriptfont3=\mathsubsubext
\textfont\scrfam=\mathscr
 \scriptfont\scrfam=\mathsubscr
 \scriptscriptfont\scrfam=\mathsubsubscr
\textfont\frfam=\mathfr
 \scriptfont\frfam=\mathsubfr
 \scriptscriptfont\frfam=\mathsubsubfr
\textfont\euexfam=\matheuex
 \scriptfont\euexfam=\mathsubsym 
 \scriptscriptfont\euexfam=\mathsubsubsym
\textfont\eqfam=\teneq

\mathcode`0="7130
\mathcode`1="7131
\mathcode`2="7132
\mathcode`3="7133
\mathcode`4="7134
\mathcode`5="7135
\mathcode`6="7136
\mathcode`7="7137
\mathcode`8="7138
\mathcode`9="7139
\mathcode`+="2\frfam@2B
\mathcode`-="2\frfam@2D
\mathcode`!="0\frfam@21
\mathcode`(="4\frfam@28 \delcode`(="\frfam@28300
\mathcode`)="5\frfam@29 \delcode`)="\frfam@29301
\mathcode`[="4\frfam@5B \delcode`[="\frfam@5B302
\mathcode`]="5\frfam@5D \delcode`]="\frfam@5D303
\mathcode`=="3\frfam@3D
\mathchardef\intop="1\euexfam@52
\mathchardef\ointop="1\euexfam@48
\mathchardef\coprod="1\euexfam@60
\mathchardef\prod="1\euexfam@51
\mathchardef\sum="1\euexfam@50
\mathchardef\braceld="\euexfam@7A \mathchardef\bracerd="\euexfam@7B
\mathchardef\bracelu="\euexfam@7C \mathchardef\braceru="\euexfam@7D
\mathchardef\infty="0\euexfam@31
\mathchardef\nearrow="3\euexfam@25
\mathchardef\searrow="3\euexfam@26
\mathchardef\nwarrow="3\euexfam@2D
\mathchardef\swarrow="3\euexfam@2E
\mathchardef\Leftrightarrow="3\euexfam@2C
\mathchardef\Leftarrow="3\euexfam@28
\mathchardef\Rightarrow="3\euexfam@29
\mathchardef\leftrightarrow="3\euexfam@24 \mathcode`\^^W="3\euexfam@24
\mathchardef\leftarrow="3\euexfam@20 \let\gets=\leftarrow \mathcode`\^^X="3\euexfam@20
\mathchardef\rightarrow="3\euexfam@21 \let\to=\rightarrow \mathcode`\^^Y="3\euexfam@21
\mathchardef\leftharpoonup="3\euexfam@18
\mathchardef\leftharpoondown="3\euexfam@19
\mathchardef\rightharpoonup="3\euexfam@1A
\mathchardef\rightharpoondown="3\euexfam@1B
\mathchardef\Relbar="3\eqfam@3D % the old = to match \Arrows
\mathchardef\Gamma="7100
\mathchardef\Delta="7101
\mathchardef\Theta="7102
\mathchardef\Lambda="7103
\mathchardef\Xi="7104
\mathchardef\Pi="7105
\mathchardef\Sigma="7106
\mathchardef\Upsilon="7107
\mathchardef\Phi="7108
\mathchardef\Psi="7109
\mathchardef\Omega="710A
\mathchardef\leq="3\scrfam@14
\mathchardef\geq="3\scrfam@15
\mathchardef\Re="0\scrfam@3C
\mathchardef\Im="0\scrfam@3D
\mathchardef\aleph="0\scrfam@40
\def\uparrow{\delimiter"3\euexfam@22378 } \mathcode`\^^K="3\euexfam@22
\def\downarrow{\delimiter"3\euexfam@23379 } \mathcode`\^^A="3\euexfam@23
\def\updownarrow{\delimiter"3\euexfam@6C33F }
\def\Uparrow{\delimiter"3\euexfam@2A37E }
\def\Downarrow{\delimiter"3\euexfam@2B37F }
\def\Updownarrow{\delimiter"3\euexfam@6D377 }
\def\rbrace{\delimiter"5\scrfam@67\euexfam@09 } \let\}=\rbrace
\def\lbrace{\delimiter"4\scrfam@66\euexfam@08 } \let\{=\lbrace
\def\vert{\delimiter"\scrfam@6A30C }
\def\backslash{\delimiter"\scrfam@6E30F }
\let\varsigma=\sigma \let\varrho=\rho % euler doesn't have these

\rm

\catcode`@=12

\endinput
